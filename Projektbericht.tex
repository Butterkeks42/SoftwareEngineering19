\documentclass[a4paper, 12pp]{article}
\usepackage[german]{babel}
\usepackage[utf8]{inputenc}
\usepackage[left=25mm,
            right=25mm,
            top=25mm,
            bottom=25mm,
            marginparsep=5mm,
            marginparwidth=40mm,
            ]{geometry}
\usepackage{float}
\usepackage{graphicx}
\graphicspath{ {./Bilder/} }

\title{\textbf{Requirements Specifications}}
\author{ Yvonne Schießl \\
		Anna Schmidbauer\\
		Bernhard Wildangel}
\date{}
\begin{document}
\maketitle
\tableofcontents
\newpage
\section{Abstract}
\subsection{Szenario ohne Atlas}

\section{Folien bzw. Pitch}

\begin{figure}[H] 
\centering
	\fbox{\begin{minipage}{13cm} 
%	   \includegraphics[width=12cm]{Folie1.JPG} 
	\end{minipage}}

	\vspace{0.3cm}

	\fbox{\begin{minipage}{13cm}  
%		\includegraphics[width=12cm]{Folie2.JPG} 
	\end{minipage}}

	\vspace{0.3cm}

	\fbox{\begin{minipage}{13cm} 
%		\includegraphics[width=12cm]{Folie3.JPG} 
	\end{minipage}}
	
Yvonne Schießl, Anna Schmidbauer, Bernhard Wildangel
	
\end{figure}

\section{Produktidee}

{\large Name: Atlas \\
Vision: Geocaching auf einem neuen Level}\\ \\
{\large Funktionen:}\\
\begin{itemize}
	\item Strecken laufen
	\item Strecken erstellen (nur erfahrene Benutzer)
	\item Gruppen erstellen
	\begin{itemize}
		\item mit Gruppe Strecken laufen
		\item mit Gruppe chatten
	\end{itemize}
	\item Herausforderungen senden
	\item Achievements erhalten
	\item extra Strecken kaufen
	\item Bewertung abgeben 
\end{itemize}
\bigskip
{\large Vorteile für Anwender:}\\
\begin{itemize}
	\item Kennenlernen der neuen Stadt/Umgebung
	\item Freunde mit gleichen Interessen finden
	\item Sport/frische Luft
\end{itemize}
\bigskip
{\large Zielgruppen:}\\
\begin{itemize}
	\item Familien
	\item Studenten
	\item Berufstätige
	\item Rentner
\end{itemize}
{\large Einschränkungen und Randbedingungen:}\\
\begin{itemize}
	\item vorerst nur auf Deutsch
	\item nur Android
	\item nur wenige vorerstellte Strecken
\end{itemize}

\section{Managementvorlage}
Nun stellen Sie sich sicher die Frage: Warum sollte man in so etwas investieren? Ist eine App für Geocaching wirklich ein Vorhaben Erfolgsaussichten? Wer sollte denn so etwas brauchen?\newline
Ein Beispiel: Sie ziehen in eine neue Stadt, eine Situation, die bestimmt jeder schon einmal erlebt hat. Nun müssen Sie sich in einer neuen Umgebung zurechtfinden, die Stadt kennenlernen, wollen vielleicht ein paar der örtlichen Sehenswürdigkeiten ansehen. Und natürlich wollen Sie bestimmt auch neue Leute treffen, immerhin kennen Sie ja noch niemanden und da ist es schwer so ganz allein zurechtzukommen.
Und genau hier kommen wir ins Spiel!\newline
Mit Atlas können Sie spielend leicht eine Stadt kennen lernen, indem Sie verschiedene Geocaching Routen ablaufen und sich dabei sogar aussuchen, ob Sie eher Routen wählen, die Touristenattraktionen beinhalten, einfach gemütliche Spaziergänge durch die Innenstadt machen oder sogar entlegenere Winkel entdecken. Da die Strecken meist von Einheimischen erstellt werden, können Sie so auch sofort die Hotspots der Ortsansässigen finden und neue Leute treffen.\newline
Oder aber Sie nutzen direkt unsere integrierte Gruppenfunktion, um Leute in Ihrer Nähe und so auch Menschen zu finden, mit denen Sie sofort eine gleiche Interesse teilen. Auch können Sie die Gruppen nutzen, um sich einfach und unkompliziert mit Menschen zum gemeinsamen Geocaching zu verabreden und so noch mehr über die Stadt zu erfahren und Zeit mit Anderen zu verbringen.\newline
Unsere App wird auch den Tourismus fördern, indem Städte und Gemeinden die Möglichkeit haben, anregende Geocaching Routen zu erstellen, die beispielsweise besonders schöne Wanderwege oder Denkmäler enthalten. Auf diese Weise können Touristen angeregt werden, die Ort zu besuchen und dort aufgrund der vielfältigen Entdeckungstouren möglichst lange zu verweilen, wodurch die Einnahmen der Gemeinden  vergrößert werden.\newline
Jedoch können nicht nur Auswärtige einen Nutzen aus unserem Produkt ziehen, auch Ortsansässige können neben den bereits genannten Punkten davon profitieren.\newline
Rentner oder ältere Menschen, die auf der Suche nach einem gemütlichen Sonntagsspaziergang sind, können durch das benutzerfreundliche Menü und die leicht verständliche Schwierigkeitsbewertung ganz einfach Strecken finden, die ihrer Vitalität und Ausdauer entsprechen.
Arbeitnehmer, die kurz in der Mittagspause etwas Abwechslung und sportliche Betätigung wollen, können durch die praktischen Zeitangaben jeder Strecke, genau die richtige für ihr Zeitfenster finden und so das Risiko von zu viel Stress oder sogar eines Burnouts mindern.\newline
Für Familien und junge Erwachsene, die nach etwas mehr Spannung und Action in ihrer Freizeit suchen, gibt es auch schwere Strecken mit einer Länge von bis zu 5 Stunden, die sogar Rätsel beinhalten können. So kann man sich den ganzen Tag austoben und auch sein grauen Zellen etwas fordern, um dann abends glücklich und zufrieden nach einem Tag im Grünen nach Hause zu kommen.\newline
Der Erfolg der App wäre also schon mal sicher. Aber halt, die für Sie wichtigste Frage bleibt ja: Wie kann man damit Geld verdienen? Ganz einfach, mit In-App Käufen!\newline
Es wird ein kostenloses Startpaket mit verschiedenen Strecken aus der Region des Nutzers geben, alle weiteren Strecken in verschiedenen Schwierigkeiten, Längen, höherem Rätselanteil oder andere Orte, können kostenpflichtig im Shop erworben werden. Außerdem werden wir für Städte und Gemeinden, die als eine unserer Hauptprofitierenden gelten, einen Service für die Streckenerstellung einführen. So können sie auf unsere Erfahrungen beim Bau von Geocaching-Strecken zugreifen und müssen nicht selbst Leute damit beauftragen und wir erhalten mit jeder erstellten Strecke Geld.

Yvonne Schießl

\section{Projektauftrag}
\begin{figure}[H] 
	\begin{minipage}{17cm} 
%	   \includegraphics[page = 1, width=17cm]{Projektauftrag.pdf} 
	\end{minipage}
\end{figure}

\begin{figure}[H] 
	\begin{minipage}{17cm} 
%	   \includegraphics[page = 2, width=17cm]{Projektauftrag.pdf} 
	\end{minipage}
\end{figure}

Yvonne Schießl

\section{Umsetzbarkeit}
\section{Stakeholder-Analyse}

\begin{center}


\begin{tabular}{|p{2cm}|p{2cm}|p{2cm}|p{2cm}|p{2cm}|p{2cm}|}
\hline
Name/Rolle & Beschreibung & konkreter Vertreter & Wissens- gebiete & Verfügbar- keit & Begründung\\ \hline
Familien & Benutzer des Systems & Hr.Müller M@oth.de Tel:7563& wenig Erfahrung mit Smartphone & 30\% verfügbar & kurze und einfache Strecken für Kinder\\ \hline
Senioren & Benutzer des Systems & Hr.Schmidt S@oth.de Tel:7336 & wenig Erfahrung mit Smartphone & 10\% verfügbar & einfache Bedienung/ Angabe der Schwierigkeit nötig\\ \hline
junge Menschen/ Berufstätige & Benutzer des Systems & Fr.Bauer B@oth.de Tel:3960& viel Erfahrung mit Smartphone & 40\% verfügbar & wenig Zeit\\ \hline
Gemeinden & Verkauf an Gemeinden & Hr.Schmidt S@oth.de Tel:3724& kennt Umgebung/ gute Locations & 20\% verfügbar & Angebot, Strecken für Gemeinden zu erstellen\\ \hline
Entwickler & Entwickler des Systems & Hr.Hofbauer H@oth.de Tel:1346& sehr gute Programmier- kentnisse & 60\% verfügbar & Entwickler setzen Projekt um\\ \hline
Service & Anbieter von Hilfe/ Wartung & Hr.Müller M@oth.de Tel:8356& IT- Spezialist & 5\% verfügbar & Problem- beheber\\ \hline
Anbieter der Zahlungs- methoden & für in App Käufe & Fr.Erhard E@oth.de Tel:9543 & kennt Zahlungs- methoden & 5\% verfügbar & für Strecken- käufe\\ \hline
Marketing & Werbung für App & Hr.Müller M@oth.de Tel:3688 & Marketing Spezialist & 40\% verfügbar & Werbung in Gemeinden\\ \hline
Strecken- ersteller & Benutzer, die Strecken erstellen können & Hr.Meier M@oth.de Tel:3745& erfahrene Strecken- läufer & 20\% verfügbar & einfacher Strecken- editor nötig\\ \hline
\hline

\end{tabular}
\end{center}

\section{Systemkontext}

\begin{figure}[H] 
\centering
	\fbox{\begin{minipage}{13cm} 
%	   \includegraphics[width=12cm]{Kontextdiagramm.png} 
	\end{minipage}}

	
\end{figure}

\section{Anforderungen}
\subsection{Funktionale Anforderungen}
{\Large Strecken laufen}\\
\begin{itemize}
\item Eingabe
	\begin{itemize}
	\item Start-button
	\end{itemize}
\item Verarbeitung
	\begin{itemize}
	\item Zeit und GPS-Daten werden aufgenommen
	\item Lösen der Rätsel an Zwischenstationen
	\item Prüfe, ob Lösung richtig, wenn nicht, nochmal versuchen
	\end{itemize}
\item Ausgabe
	\begin{itemize}
	\item Koordinaten der nächsten Location
	\end{itemize}
\end{itemize}


\bigskip
{\Large Strecken erstellen}\\
\begin{itemize}
\item Eingabe
	\begin{itemize}
	\item Schwierigkeit, Zeit, Beschreibung, Koordinaten der Zwischen-/ 						Ziellocation, Rätsel
	\end{itemize}
\item Verarbeitung
	\begin{itemize}
	\item auf Vollständigkeit prüfen
	\item wenn nicht vollständig, werden die fehlenden Daten eingetragen
	\item auf sinnvolle Eingaben prüfen
	\item wenn nicht sinnvoll, werden die Daten erneut eingetragen
	\end{itemize}
\item Ausgabe
	\begin{itemize}
	\item erstellte Strecke
	\end{itemize}
\end{itemize}

\bigskip
{\Large Gruppen erstellen}\\
\begin{itemize}
\item Eingabe
	\begin{itemize}
	\item Name, Mitglieder
	\end{itemize}
\item Verarbeitung
	\begin{itemize}
	\item Mitglieder werden hinzugefügt
	\end{itemize}
\item Ausgabe
	\begin{itemize}
	\item erstellte Gruppe
	\end{itemize}
\end{itemize}

\bigskip
{\Large Herausforderungen senden}\\
\begin{itemize}
\item Eingabe
	\begin{itemize}
	\item Namen der Gegner, Strecke
	\end{itemize}
\item Verarbeitung
	\begin{itemize}
	\item Herausforderungen annehmen oder nicht
	\end{itemize}
\item Ausgabe
	\begin{itemize}
	\item Herausforderung angenommen oder nicht
	\end{itemize}
\end{itemize}

\bigskip
{\Large Achievements erhalten}\\
\begin{itemize}
\item Eingabe
	\begin{itemize}
	\item User erfüllt bedingung eines Achievements
	\end{itemize}
\item Verarbeitung
	\begin{itemize}
	\item Achievement wird freigeschaltet
	\end{itemize}
\item Ausgabe
	\begin{itemize}
	\item Punkte, Icon
	\end{itemize}
\end{itemize}

\bigskip
{\Large Strecken kaufen}\\
\begin{itemize}
\item Eingabe
	\begin{itemize}
	\item Strecke/ Streckenpaket
	\end{itemize}
\item Verarbeitung
	\begin{itemize}
	\item User zahlt Artikel
	\item User kann durch Punkte Rabatt bekommen
	\item wenn bezahlt, werden gekaufte Strecken freigeschaltet
	\end{itemize}
\item Ausgabe
	\begin{itemize}
	\item neue Strecken
	\end{itemize}
\end{itemize}

\bigskip
{\Large Bewertung abgeben}\\
\begin{itemize}
\item Eingabe
	\begin{itemize}
	\item Text, Bewertung der Schwierigkeit
	\end{itemize}
\item Verarbeitung
	\begin{itemize}
	\item Bewertung senden
	\end{itemize}
\item Ausgabe
	\begin{itemize}
	\item Bewertung
	\end{itemize}
\end{itemize}

\bigskip
{\Large Chat benutzen}\\
\begin{itemize}
\item Eingabe
	\begin{itemize}
	\item Nachricht
	\end{itemize}
\item Verarbeitung
	\begin{itemize}
	\item Nachrichten senden / anzeigen / löschen
	\end{itemize}
\item Ausgabe
	\begin{itemize}
	\item Nachricht mit Ersteller, Zeit
	\end{itemize}
\end{itemize}
\subsection{Nichtfunktionale Anforderungen}
\subsubsection{Technische Anforderungen}
• Das System muss mit Java entwickelt werden.\newline
• Der Speicherbedarf darf 1,5 GB nicht übersteigen.\newline
• Der Speicherbedarf im RAM darf nicht 150 MByte übersteigen.\newline
• Das System muss auf allen Smartphones mit einem Android Betriebssystem Version 8 oder höher laufen.
\subsubsection{Ergonomische Anforderungen}
• Die Benutzerführung erfolgt in deutsch.\newline
• Die Schriftgröße soll wählbar zwischen normal (12pt) und groß (15pt) sein.\newline
• Das System soll sich das User-Verhalten bei der Suche nach Strecken merken und dementsprechend passende Strecken bereits vorschlagen.
\subsubsection{Anforderungen an die Dienstqualität:}
• Das System soll bei einer durchschnittliche Internetgeschwindigkeit von 13.000 Kilobit/s maximal 2 Sekunden benötigen, um den s User einzuloggen.\newline
• Das System soll bei einer durchschnittliche Internetgeschwindigkeit von 13.000 Kilobit/s maximal 5 Sekunden benötigen, um auf die Datenbank der Strecken zuzugreifen.\newline
• Das vom System genutzte GPS darf nicht an Genauigkeit verlieren.
\subsubsection{Rechtliche Anforderungen}
• Der Kunde leistet ein Viertel der vertraglich für die Systementwicklung vereinbarten Summe im Vorraus.\newline
• Die deutschen Datenschutzrichtlinien müssen erfüllt sein.\newline
• Die Datenbank mit den Daten der User ist bestmöglich nach aktuellem Kenntnisstand der IT-Security zu schützen.\newline
• Die Datenbanken werden vierteljährlich gewartet.
\subsubsection{Weitere Arten Nichtfunktionaler Anforderungen}
Anforderungen an die Zuverlässigkeit:\newline
• Das System soll eine Absturzrate  $\leq \ $1\% haben.

Yvonne Schießl

\subsection{Abnahmekriterien}
• Es ist möglich sich ein User-Konto einzurichten. Hierfür enthält die App auch eine Log-In Funktion\newline
• Es ist möglich aus einer Datenbank Strecken auszuwählen und diese mit der App abzulaufen.\newline
• Es ist für bestimmte Nutzer möglich selber Strecken zu erstellen und diese in die Datenbank hochzuladen.

Yvonne Schießl

\section{Risiken}
Natürlich birgt dieses Projekt wie jedes andere auch seine Risiken.\newline
So gibt es zum Beispiel immer das große Risiko der Akzeptanz. Sind die Leute dazu bereit unsere App zu nutzen? Wir glauben, ja! Denn unsere Idee vereint Lösungen zu vielen Problematiken, denen sich fast jeder einmal bereits gegenüber sah, wie beispielsweise einen neuen Ort zu entdecken oder Gleichgesinnte zu treffen. Wir planen ebenfalls durch eine Werbecampagne mit angesehenen YouTubern die allgemeine Begeisterung für Geocaching vor allem bei Jugendlichen und jungen Erwachsenen zu steigern.\newline
Aber auch wenn die App genutzt wird, ist das keine Garantie dafür, damit auch Geld zu verdienen, da das Ausgeben von Geld für Strecken optional ist. Dennoch sind wir davon überzeugt, dass das Geocaching die Leute so fesseln wird, dass sie bald dazu bereit sein werden Geld für mehr und anspruchsvollere Strecken auszugeben. Auch setzen wir auf einen eher niedrigeren Preis für Strecken und häufige Angebote für Pakete, um so zwar weniger Geld mit einem einzelnen Verkauf, aber mehr Verkäufe insgesamt zu machen.\newline
Wenn man nun im Internet mal danach sucht, wird man feststellen, dass es bereits ein paar Apps für Geocaching gibt. Warum sollten die Kunden nun unser Produkt wählen und nicht eines der Konkurrenz? Zum einen setzen wir auf eine sehr benutzerfreundliche Oberfläche, um so auch Kinder oder ältere Menschen ald Nutzer zu gewinnen. Zum anderen legt kein anderes System so viel Wert auf die Interaktion der Nutzer untereinander wie wir. Durch unseren starken Fokus auf den sozialen Aspekt erfüllen wir ein Grundbedürfnis des Menschen und können auch eine emotionale Bindung zur unserer App herstellen, indem sie die Möglichkeit erhalten neue Freunde zu finden, mit ihnen zu chatten, sich zu treffen oder auch andere herauszufordern. Alles Punkte die für unsere Anwendung und nicht die der Konkurrenz sprechen.\newline
Doch woher kommen unsere Strecken? Wenn keine neuen dazu kommen wird das Geocaching doch recht schnell langweilig. Hier nutzen wir das altbekannte System, dass Nutzer für Nutzer erstellen. Nämlich wird es spezielle Accounts für erfahrenere Nutzer geben, die die Möglichkeit bekommen, selber Strecken zu bauen und diese zu veröffentlichen. Und für den Fall dass es zu wenig Streckenersteller geben sollte, oder auch in der Anfangsphase der App, gibt es immer noch von uns angestellte Streckenersteller, die zusätzlich Geocaching Routen bauen.\newline
Ein gewisses Restrisiko bleibt natürlich immer, dennoch haben wir uns darum bemüht sämtliche Risiken nach besten Möglichkeiten zu beseitigen.

Yvonne Schießl

\section{Kano-Modell}
\section{Use-Cases}
\subsection{Use-Case-Diagramm}

\begin{figure}[H] 
\centering
	\fbox{\begin{minipage}{16cm} 
%	   \includegraphics[width=15cm]{atlas_usecasediagramm.png} 
	\end{minipage}}
\end{figure}
Yvonne Schießl, Anna Schmidbauer, Bernhard Wildangel

\subsection{Use-Case-Beschreibungen}
\section{Aktivitätsdiagramm}

{\Large Authentifizierung}
Anna Schmidbauer
\begin{figure}[H] 
\centering
	\fbox{\begin{minipage}{16cm} 
%	   \includegraphics[width=15cm]{authentifizierung_activity.png} 
	\end{minipage}}
\end{figure}


{\Large Bewertung abgeben}
Anna Schmidbauer
\begin{figure}[H] 
\centering
	\fbox{\begin{minipage}{16cm} 
%	   \includegraphics[width=15cm]{bewertung_activity.png} 
	\end{minipage}}
\end{figure}

{\Large Strecke erstellen}
Anna Schmidbauer
\begin{figure}[H] 
\centering
	\fbox{\begin{minipage}{16cm} 
%	   \includegraphics[width=15cm]{streckeerstellen_activity.png} 
	\end{minipage}}
\end{figure}

Erklärung:\\
Beim Erstellen einer neuen Strecke hat man die Möglichkeit, zusätzlich zum Zielort mehrere Zwischenorte hinzuzufügen. An diesen Orten bekommt man die Koordinaten für die nächste Station. Wahlweise lassen sich für die Stationen Rätsel hinzufügen, die man erst lösen muss um die nächsten Koordinaten zu erhalten. Der Zielort, Beschreibung, Schwierigkeit und Zeitaufwand müssen hinzugefügt werden.\\

{\Large Achievement erhalten}
Yvonne Schießl
\begin{figure}[H] 
\centering
	\fbox{\begin{minipage}{16cm} 
%	   \includegraphics[width=15cm]{AchievmentErhalten_Aktivität.JPG} 
	\end{minipage}}
\end{figure}

{\Large Chat benutzen}
Yvonne Schießl
\begin{figure}[H] 
\centering
	\fbox{\begin{minipage}{16cm} 
%	   \includegraphics[width=15cm]{ChatBenutzen_Aktivität.JPG} 
	\end{minipage}}
\end{figure}

{\Large Herausforderung senden}
Yvonne Schießl
\begin{figure}[H] 
\centering
	\fbox{\begin{minipage}{16cm} 
%	   \includegraphics[width=15cm]{HerausforderungSenden_Aktivität.JPG} 
	\end{minipage}}
\end{figure}

{\Large Strecke laufen}
Yvonne Schießl
\begin{figure}[H] 
\centering
	\fbox{\begin{minipage}{16cm} 
%	   \includegraphics[width=15cm]{StreckeLaufen_Aktivität.JPG} 
	\end{minipage}}
\end{figure}

\section{Zustandsdiagramme}

{\Large Authentifizierung}
Anna Schmidbauer
\begin{figure}[H] 
\centering
	\fbox{\begin{minipage}{16cm} 
%	   \includegraphics[width=15cm]{authentifizierung_state.png} 
	\end{minipage}}
\end{figure}


{\Large Strecke erstellen}
Anna Schmidbauer
\begin{figure}[H] 
\centering
	\fbox{\begin{minipage}{16cm} 
%	   \includegraphics[width=15cm]{streckeerstellen_state.png} 
	\end{minipage}}
\end{figure}

Erklärung:\\
Beim Erstellen einer neuen Strecke hat man die Möglichkeit, zusätzlich zum Zielort mehrere Zwischenorte hinzuzufügen. An diesen Orten bekommt man die Koordinaten für die nächste Station. Wahlweise lassen sich für die Stationen Rätsel hinzufügen, die man erst lösen muss um die nächsten Koordinaten zu erhalten. Der Zielort, Beschreibung, Schwierigkeit und Zeitaufwand müssen hinzugefügt werden.\\

{\Large Achievement erhalten}
Yvonne Schießl
\begin{figure}[H] 
\centering
	\fbox{\begin{minipage}{16cm} 
%	   \includegraphics[width=16cm]{AchievementErhalten_Zustand.JPG} 
	\end{minipage}}
\end{figure}

\section{Klassendiagramme}


\begin{figure}[H] 
\centering
	\fbox{\begin{minipage}{16cm} 
%	   \includegraphics[width=15cm]{classdiagramm_anna.png} 
	\end{minipage}}
\end{figure}
Anna Schmidbauer\\
Erklärung:\\
Es gibt zwei Arten von Benutzern: den normalen User und den Streckenersteller, der alle Rechte eines normalen Users hat, aber zusätzlich Strecken erstellen darf. Eine Strecke hat mehrere Bewertungen und Rätsel und ist Teil des Shops, in dem man manche Strecken kaufen kann. Ein Event mit mehreren Teilnehmern hat einen Chat, in dem mehrere Nachrichten stehen können. Es gibt zwei Arten von Events: ein Gruppenevent, bei dem man eine Strecke mit mehreren Leuten läuft, und die Wettbewerbsfunktion, bei der man anderen Usern eine Herausforderung schicken kann.\\

\begin{figure}[H] 
\centering
	\fbox{\begin{minipage}{16cm} 
%	   \includegraphics[width=16cm]{Klassendiagram_Yvonne.png} 
	\end{minipage}}
\end{figure}
Yvonne Schießl\\
Erklärung:\\
Es gibt zwei Arten von Benutzern: den normalen User und den Streckenersteller, der alle Rechte eines normalen Users hat, aber zusätzlich Strecken erstellen darf. Eine Strecke kann mehrere Bewertungen von Usern haben, jede Bewertung braucht aber eine zugeordnete Strecke. Artikel ist eine Klasse die die Strecke um einen Preis und eine Beliebtheit erweitert, sodass im Shop Artikel angezeigt und von Usern gekauft werden können. Es gibt zwei verschiedene Arten von Events, die jedoch im Grunde die gleichen Eigenschaften teilen und sich nur in einigen Spezifikationen unterschieden: ein Gruppenevent, bei dem man eine Strecke mit mehreren Leuten läuft, und die Wettbewerbsfunktion, bei der man anderen Usern eine Herausforderung schicken kann. In Chats kann man Nachrichten an andere User schicken bzw. Nachrichten von diesen empfangen, jede Nachricht muss jedoch Teil eines bestimmten Chats mit einem anderen User sein und kann nicht ohne einen Chat existieren.\\

\begin{figure}[H] 
\centering
	\fbox{\begin{minipage}{16cm} 
%	   \includegraphics[width=15cm]{classdiagramm_group.png} 
	\end{minipage}}
\end{figure}
Yvonne Schießl, Anna Schmidbauer, Bernhard Wildangel\\
Erklärung:\\
Es gibt zwei Arten von Benutzern: den normalen User und den Streckenersteller, der alle Rechte eines normalen Users hat, aber zusätzlich Strecken erstellen darf. Die Anzahl der erstellten Strecken berechnet sich aus der Liste der erstellten Strecken und die Anzahl der gelaufenen Strecken berechnet sich aus der Liste der gelaufenen Strecken. Die Punkte die ein User hat, berechnen sich aus den gewonnenen Wettbewerben und seinen Achievements.\\
Eine Strecke hat mehrere Bewertungen, einen Zielpunkt und kann mehrere Zwischenpunkte enthalten, die wiederum Rätsel haben können. Außerdem kann eine Strecke einen Preis haben, wenn die Strecke im Shop zu kaufen ist. Die Schwierigkeit ergibt sich aus den Angaben, die der Ersteller anfangs gemacht hat, und aus den Bewertungen, die die User erstellen.\\
Es gibt zwei Arten von Events: ein Gruppenevent, bei dem man eine Strecke mit mehreren Leuten läuft, und die Wettbewerbsfunktion, bei der man anderen Usern eine Herausforderung schicken kann. Eine Gruppe aus Teilnehmern ist immer Teil eines Events, kann aber auch ohne Event existieren. Eine Gruppe hat auch einen Chat, welcher aus mehreren Nachrichten besteht, mindestens zwei Teilnehmer hat und auch ohne Gruppe existieren kann.\\


\section{Sequenzdiagramme}

{\Large Strecke erstellen}
Anna Schmidbauer
\begin{figure}[H] 
\centering
	\fbox{\begin{minipage}{16cm} 
%	   \includegraphics[width=15cm]{streckeerstellen_seq.png} 
	\end{minipage}}
\end{figure}

Erklärung:\\
Die Strecken werden im Streckeneditor erstellt. Beim Erstellen einer neuen Strecke hat man die Möglichkeit, zusätzlich zum Zielort mehrere Zwischenorte hinzuzufügen. An diesen Orten bekommt man die Koordinaten für die nächste Station. Wahlweise lassen sich für die Stationen Rätsel hinzufügen, die man erst lösen muss um die nächsten Koordinaten zu erhalten. Der Zielort, Beschreibung, Schwierigkeit und Zeitaufwand müssen hinzugefügt werden. Wenn alles bestätigt wurde, wird in der Strecken-Datenbank eine neue Strecke erstellt.\\

{\Large Achievement erhalten}
Yvonne Schießl
\begin{figure}[H] 
\centering
	\fbox{\begin{minipage}{16cm} 
%	   \includegraphics[width=16cm]{AchievementErhalten_Sequenz.JPG} 
	\end{minipage}}
\end{figure}

\section{Komponenten- und Verteilungsdiagramm}
\subsection{Komponentendiagramm}

Anna Schmidbauer
\begin{figure}[H] 
\centering
	\fbox{\begin{minipage}{16cm} 
%	   \includegraphics[width=15cm]{component.png} 
	\end{minipage}}
\end{figure}

\subsection{Verteilungsdiagramm}
\section{Fazit}
Wir sehen ein sehr großes Potential in unserer App Atlas, da wir das Konzept von Geocaching auf ein neues Level bringen. Zum einen sprechen wir eine große Zielgruppe an und zum anderen bieten wir eine tolle Möglichkeit, neue Freunde und eine neue Umgebung kennenzulernen. Wir machen es möglich, dass so gut wie jede Altersgruppe Atlas benutzen kann, indem wir mehrere Strecken mit verschiedenen Schwierigkeitsstufen und Zeitaufwand anbieten.\\
Atlas wird vor allem von den Menschen genutzt werden, die Spaß an der Bewegung haben und neue Orte entdecken wollen. Mit unserer Gruppenfunktion bringen wir die Menschen zusammen und übertragen die Begeisterung für Geocaching auf neue Benutzer.\\
Auch die Finanzierung ist gesichert, da wir eng mit Gemeinden zusammenarbeiten wollen, damit sie Routen durch den Ort erstellen können, um den Tourismus zu fördern. Außerdem bieten wir unseren Kunden zusätzliche Strecken im Shop an, die sie dann einzeln oder in Paketen erwerben können.\\
Sobald sich Atlas bei den Benutzern etabliert hat, wollen wir auch in der Zukunft unsere App verbessern. Wir planen, die Anzahl der Strecken, auch mithilfe unserer Streckenersteller zu erweitern, um so unseren Radius zu erweitern. Auch durch die Zusammenarbeit mit Gemeinden wollen wir neue Kunden bekommen. Außerdem soll Atlas in der Zukunft in weiteren Sprachen verfügbar sein, wie zum Beispiel Englisch, um auch nicht deutsch-kundige Menschen und Touristen anzusprechen. Auch indem wir auf die Forderungen unserer User eingehen, wollen wir unsere App verbessern und neue Features hinzufügen.\\
Wir sind zuversichtlich, dass Atlas bei unseren Benutzern große Beliebtheit erlangen wird und dass wir auch schnell neue Kunden erschließen können.\\
\end{document}
