\documentclass[a4paper, 12pp]{article}
\usepackage[german]{babel}
\usepackage[utf8]{inputenc}
\usepackage[left=25mm,
            right=25mm,
            top=25mm,
            bottom=25mm,
            marginparsep=5mm,
            marginparwidth=40mm,
            ]{geometry}
\usepackage{float}
\usepackage{graphicx}
\graphicspath{ {./Bilder/} }

\title{\textbf{Requirements Specifications}}
\author{ Yvonne Schießl \\
		Anna Schmidbauer\\
		Bernhard Wildangel}
\date{}
\begin{document}
\maketitle
\tableofcontents
\newpage
\section{Abstract}
\subsection{Szenario ohne Atlas}

\section{Folien bzw. Pitch}

\begin{figure}[H] 
\centering
	\fbox{\begin{minipage}{13cm} 
%	   \includegraphics[width=12cm]{Folie1.JPG} 
	\end{minipage}}

	\vspace{0.3cm}

	\fbox{\begin{minipage}{13cm}  
%		\includegraphics[width=12cm]{Folie2.JPG} 
	\end{minipage}}

	\vspace{0.3cm}

	\fbox{\begin{minipage}{13cm} 
%		\includegraphics[width=12cm]{Folie3.JPG} 
	\end{minipage}}
	
\end{figure}

\section{Produktidee}

{\large Name: Atlas \\
Vision: Geocaching auf einem neuen Level}\\ \\
{\large Funktionen:}\\
\begin{itemize}
	\item Strecken laufen
	\item Strecken erstellen (nur erfahrene Benutzer)
	\item Gruppen erstellen
	\begin{itemize}
		\item mit Gruppe Strecken laufen
		\item mit Gruppe chatten
	\end{itemize}
	\item Herausforderungen senden
	\item Achievements erhalten
	\item extra Strecken kaufen
	\item Bewertung abgeben 
\end{itemize}
\bigskip
{\large Vorteile für Anwender:}\\
\begin{itemize}
	\item Kennenlernen der neuen Stadt/Umgebung
	\item Freunde mit gleichen Interessen finden
	\item Sport/frische Luft
\end{itemize}
\bigskip
{\large Zielgruppen:}\\
\begin{itemize}
	\item Familien
	\item Studenten
	\item Berufstätige
	\item Rentner
\end{itemize}
{\large Einschränkungen und Randbedingungen:}\\
\begin{itemize}
	\item vorerst nur auf Deutsch
	\item nur Android
	\item nur wenige vorerstellte Strecken
\end{itemize}

\section{Managementvorlage}
Nun stellen Sie sich sicher die Frage: Warum sollte man in so etwas investieren? Ist eine App für Geocaching wirklich ein Vorhaben Erfolgsaussichten? Wer sollte denn so etwas brauchen?\newline
Ein Beispiel: Sie ziehen in eine neue Stadt, eine Situation, die bestimmt jeder schon einmal erlebt hat. Nun müssen Sie sich in einer neuen Umgebung zurechtfinden, die Stadt kennenlernen, wollen vielleicht ein paar der örtlichen Sehenswürdigkeiten ansehen. Und natürlich wollen Sie bestimmt auch neue Leute treffen, immerhin kennen Sie ja noch niemanden und da ist es schwer so ganz allein zurechtzukommen.
Und genau hier kommen wir ins Spiel!\newline
Mit Atlas können Sie spielend leicht eine Stadt kennen lernen, indem Sie verschiedene Geocaching Routen ablaufen und sich dabei sogar aussuchen, ob Sie eher Routen wählen, die Touristenattraktionen beinhalten, einfach gemütliche Spaziergänge durch die Innenstadt machen oder sogar entlegenere Winkel entdecken. Da die Strecken meist von Einheimischen erstellt werden, können Sie so auch sofort die Hotspots der Ortsansässigen finden und neue Leute treffen.\newline
Oder aber Sie nutzen direkt unsere integrierte Gruppenfunktion, um Leute in Ihrer Nähe und so auch Menschen zu finden, mit denen Sie sofort eine gleiche Interesse teilen. Auch können Sie die Gruppen nutzen, um sich einfach und unkompliziert mit Menschen zum gemeinsamen Geocaching zu verabreden und so noch mehr über die Stadt zu erfahren und Zeit mit Anderen zu verbringen.\newline
Unsere App wird auch den Tourismus fördern, indem Städte und Gemeinden die Möglichkeit haben, anregende Geocaching Routen zu erstellen, die beispielsweise besonders schöne Wanderwege oder Denkmäler enthalten. Auf diese Weise können Touristen angeregt werden, die Ort zu besuchen und dort aufgrund der vielfältigen Entdeckungstouren möglichst lange zu verweilen, wodurch die Einnahmen der Gemeinden  vergrößert werden.\newline
Jedoch können nicht nur Auswärtige einen Nutzen aus unserem Produkt ziehen, auch Ortsansässige können neben den bereits genannten Punkten davon profitieren.\newline
Rentner oder ältere Menschen, die auf der Suche nach einem gemütlichen Sonntagsspaziergang sind, können durch das benutzerfreundliche Menü und die leicht verständliche Schwierigkeitsbewertung ganz einfach Strecken finden, die ihrer Vitalität und Ausdauer entsprechen.
Arbeitnehmer, die kurz in der Mittagspause etwas Abwechslung und sportliche Betätigung wollen, können durch die praktischen Zeitangaben jeder Strecke, genau die richtige für ihr Zeitfenster finden und so das Risiko von zu viel Stress oder sogar eines Burnouts mindern.\newline
Für Familien und junge Erwachsene, die nach etwas mehr Spannung und Action in ihrer Freizeit suchen, gibt es auch schwere Strecken mit einer Länge von bis zu 5 Stunden, die sogar Rätsel beinhalten können. So kann man sich den ganzen Tag austoben und auch sein grauen Zellen etwas fordern, um dann abends glücklich und zufrieden nach einem Tag im Grünen nach Hause zu kommen.\newline
Der Erfolg der App wäre also schon mal sicher. Aber halt, die für Sie wichtigste Frage bleibt ja: Wie kann man damit Geld verdienen? Ganz einfach, mit In-App Käufen!\newline
Es wird ein kostenloses Startpaket mit verschiedenen Strecken aus der Region des Nutzers geben, alle weiteren Strecken in verschiedenen Schwierigkeiten, Längen, höherem Rätselanteil oder andere Orte, können kostenpflichtig im Shop erworben werden. Außerdem werden wir für Städte und Gemeinden, die als eine unserer Hauptprofitierenden gelten, einen Service für die Streckenerstellung einführen. So können sie auf unsere Erfahrungen beim Bau von Geocaching-Strecken zugreifen und müssen nicht selbst Leute damit beauftragen und wir erhalten mit jeder erstellten Strecke Geld.

\section{Projektauftrag}
\section{Umsetzbarkeit}
\section{Stakeholder-Analyse}

\begin{center}


\begin{tabular}{|p{2cm}|p{2cm}|p{2cm}|p{2cm}|p{2cm}|p{2cm}|}
\hline
Name/Rolle & Beschreibung & konkreter Vertreter & Wissens- gebiete & Verfügbar- keit & Begründung\\ \hline
Familien & Benutzer des Systems & Hr.Müller M@oth.de Tel:7563& wenig Erfahrung mit Smartphone & 30\% verfügbar & kurze und einfache Strecken für Kinder\\ \hline
Senioren & Benutzer des Systems & Hr.Schmidt S@oth.de Tel:7336 & wenig Erfahrung mit Smartphone & 10\% verfügbar & einfache Bedienung/ Angabe der Schwierigkeit nötig\\ \hline
junge Menschen/ Berufstätige & Benutzer des Systems & Fr.Bauer B@oth.de Tel:3960& viel Erfahrung mit Smartphone & 40\% verfügbar & wenig Zeit\\ \hline
Gemeinden & Verkauf an Gemeinden & Hr.Schmidt S@oth.de Tel:3724& kennt Umgebung/ gute Locations & 20\% verfügbar & Angebot, Strecken für Gemeinden zu erstellen\\ \hline
Entwickler & Entwickler des Systems & Hr.Hofbauer H@oth.de Tel:1346& sehr gute Programmier- kentnisse & 60\% verfügbar & Entwickler setzen Projekt um\\ \hline
Service & Anbieter von Hilfe/ Wartung & Hr.Müller M@oth.de Tel:8356& IT- Spezialist & 5\% verfügbar & Problem- beheber\\ \hline
Anbieter der Zahlungs- methoden & für in App Käufe & Fr.Erhard E@oth.de Tel:9543 & kennt Zahlungs- methoden & 5\% verfügbar & für Strecken- käufe\\ \hline
Marketing & Werbung für App & Hr.Müller M@oth.de Tel:3688 & Marketing Spezialist & 40\% verfügbar & Werbung in Gemeinden\\ \hline
Strecken- ersteller & Benutzer, die Strecken erstellen können & Hr.Meier M@oth.de Tel:3745& erfahrene Strecken- läufer & 20\% verfügbar & einfacher Strecken- editor nötig\\ \hline
\hline

\end{tabular}
\end{center}

\section{Systemkontext}

\begin{figure}[H] 
\centering
	\fbox{\begin{minipage}{13cm} 
%	   \includegraphics[width=12cm]{Kontextdiagramm.png} 
	\end{minipage}}

	
\end{figure}

\section{Anforderungen}
\subsection{Funktionale Anforderungen}
{\Large Strecken laufen}\\
\begin{itemize}
\item Eingabe
	\begin{itemize}
	\item Start-button
	\end{itemize}
\item Verarbeitung
	\begin{itemize}
	\item Zeit und GPS-Daten werden aufgenommen
	\item Lösen der Rätsel an Zwischenstationen
	\item Prüfe, ob Lösung richtig, wenn nicht, nochmal versuchen
	\end{itemize}
\item Ausgabe
	\begin{itemize}
	\item Koordinaten der nächsten Location
	\end{itemize}
\end{itemize}


\bigskip
{\Large Strecken erstellen}\\
\begin{itemize}
\item Eingabe
	\begin{itemize}
	\item Schwierigkeit, Zeit, Beschreibung, Koordinaten der Zwischen-/ 						Ziellocation, Rätsel
	\end{itemize}
\item Verarbeitung
	\begin{itemize}
	\item auf Vollständigkeit prüfen
	\item wenn nicht vollständig, werden die fehlenden Daten eingetragen
	\item auf sinnvolle Eingaben prüfen
	\item wenn nicht sinnvoll, werden die Daten erneut eingetragen
	\end{itemize}
\item Ausgabe
	\begin{itemize}
	\item erstellte Strecke
	\end{itemize}
\end{itemize}

\bigskip
{\Large Gruppen erstellen}\\
\begin{itemize}
\item Eingabe
	\begin{itemize}
	\item Name, Mitglieder
	\end{itemize}
\item Verarbeitung
	\begin{itemize}
	\item Mitglieder werden hinzugefügt
	\end{itemize}
\item Ausgabe
	\begin{itemize}
	\item erstellte Gruppe
	\end{itemize}
\end{itemize}

\bigskip
{\Large Herausforderungen senden}\\
\begin{itemize}
\item Eingabe
	\begin{itemize}
	\item Namen der Gegner, Strecke
	\end{itemize}
\item Verarbeitung
	\begin{itemize}
	\item Herausforderungen annehmen oder nicht
	\end{itemize}
\item Ausgabe
	\begin{itemize}
	\item Herausforderung angenommen oder nicht
	\end{itemize}
\end{itemize}

\bigskip
{\Large Achievements erhalten}\\
\begin{itemize}
\item Eingabe
	\begin{itemize}
	\item User erfüllt bedingung eines Achievements
	\end{itemize}
\item Verarbeitung
	\begin{itemize}
	\item Achievement wird freigeschaltet
	\end{itemize}
\item Ausgabe
	\begin{itemize}
	\item Punkte, Icon
	\end{itemize}
\end{itemize}

\bigskip
{\Large Strecken kaufen}\\
\begin{itemize}
\item Eingabe
	\begin{itemize}
	\item Strecke/ Streckenpaket
	\end{itemize}
\item Verarbeitung
	\begin{itemize}
	\item User zahlt Artikel
	\item User kann durch Punkte Rabatt bekommen
	\item wenn bezahlt, werden gekaufte Strecken freigeschaltet
	\end{itemize}
\item Ausgabe
	\begin{itemize}
	\item neue Strecken
	\end{itemize}
\end{itemize}

\bigskip
{\Large Bewertung abgeben}\\
\begin{itemize}
\item Eingabe
	\begin{itemize}
	\item Text, Bewertung der Schwierigkeit
	\end{itemize}
\item Verarbeitung
	\begin{itemize}
	\item Bewertung senden
	\end{itemize}
\item Ausgabe
	\begin{itemize}
	\item Bewertung
	\end{itemize}
\end{itemize}

\bigskip
{\Large Chat benutzen}\\
\begin{itemize}
\item Eingabe
	\begin{itemize}
	\item Nachricht
	\end{itemize}
\item Verarbeitung
	\begin{itemize}
	\item Nachrichten senden / anzeigen / löschen
	\end{itemize}
\item Ausgabe
	\begin{itemize}
	\item Nachricht mit Ersteller, Zeit
	\end{itemize}
\end{itemize}
\subsection{Nichtfunktionale Anforderungen}
\subsubsection{Technische Anforderungen}
\subsubsection{Ergonomische Anforderungen}
\subsubsection{Anforderungen an die Dienstqualität:}
\subsubsection{Rechtliche Anforderungen}
\subsubsection{Weitere Arten Nichtfunktionaler Anforderungen}
\subsection{Abnahmekriterien}
\newpage
\section{Risiken}
\section{Kano-Modell}
\section{Use-Cases}
\subsection{Use-Case-Diagramm}
\subsection{Use-Case-Beschreibungen}
\section{Aktivitätsdiagramm}
\section{Zustandsdiagramme}
\section{Klassendiagramme}
\section{Sequenzdiagramme}
\section{Komponenten- und Verteilungsdiagramm}
\subsection{Komponentendiagramm}
\subsection{Verteilungsdiagramm}
\section{Fazit}
\end{document}
